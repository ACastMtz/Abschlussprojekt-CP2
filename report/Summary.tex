During this project we gained a lot of insights and knowledge on how neural networks work and what is possible and impossible with machine learning.
But we also learned a lot on a more general level, like how to improve your code by testing and working as a team on a software project. This chapter is a summary of these findings and it gives a short outlook on how this project could be improved in the future.  
\section{Lesson learned}
\textbf{Define a common base:} In the beginning everyone wrote their own version of what we thought was a neural network. This made it hard to compare our code because everyone structured it differently and used other cryptic names for the variables and functions. After we agreed on a common set of necessary function and a way to name variables it got much easier to understand the code of another and work as a team. \\
\textbf{Review the code:} Sometimes you get blind for your own code. You spent hours on finding a, in the end trivial, mistake. A fresh view can speed up this process and therefore it saved us a lot of time asking team mates for a code review.  \\
\textbf{Test your code helps to find and prevent bugs:} Testing forces you to cut your program in as small as possible functions,  which by itself already decreases bugs because these functions are easier to understand and you think more about the structure of your program. Also these tests verify certain functions and speed up the error search for new bugs.

\section{What next?}
In the course of this project we only implemented a simple version of a feedforward neural network with stochastic gradient descent with plenty of room for optimization. One way to improve our program would be to learn more about how matlab works and then speed up the calculations, decreasing the time needed for our network to train preparing it for more advanced data sets.  There are also many ways to tweak the stochastic gradient descent algorithm for faster learning and preventing it from getting stuck in local minima, that we put  aside in the limited time of the project but otherwise would be quite interesting. 